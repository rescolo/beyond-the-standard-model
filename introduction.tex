%instiki:category: QuantumFieldTheory
%instiki:
%instiki:***
%instiki:
%instiki:[[Beyond|Contents]]
%instiki:
%instiki:***

\chapter*{Introduction}
\label{cha:introduction} %noinstiki

\addcontentsline{toc}{chapter}{Introduction} %noinstiki

We have organized the topics in order of complexity, and, in the same
spirit than in  previous book \cite{lsm}, we have tried to write the
calculations as detailed as possible. In
Chapter 
\ref{cha:second-quantization} %noinstiki [[Chapter II]]
we included the building blocks
of quantum field theory, in Chapter 
\ref{cha:s-matrix} %noinstiki [[Chapter III]]
we introduce
the $S$--matrix in the Scr\"odinger Picture separating the kinematical
and normalization factors from the matrix element. Then the expressions
for the decay rates and cross sections are obtained. The explicit
calculation of the matrix element from the expansion of the
$S$--matrix to obtain the Feynman rules, is postponed to Chapter
\ref{chap:fr}. %noinstiki [[Chapter V]].
In Chapter 
\ref{cha:two-body-decays} %noinstiki [[Chapter IV]]
we use the Feynman
rules necessary to calculates the matrix element, and develop the
techniques associated to the squaring of the matrix element. In
Chapter 
\ref{chap:fr} %noinstiki [[Chapter V]]
we obtain the Feynman rules used in two body
decays directly from the first order expansion of the $S$--matrix in
the interaction picture. The subsequent chapters have applications of
the techniques developed to the calculation of tree-level, 
Chapter 
\ref{cha:three-body-decays} %noinstiki [[Chapter V]]
and loop processes.


This notes are based in books \cite{Maggiore:2005qv}, \cite{Mandl:1985bg}, \cite{Lahiri:2005sm}.  In each Chapter or Section the main reference used is cited. Also, we have included material developed by Juan Alberto Yepez, José David Ruiz Álvarez,
Tomas Atehortua Garcés,
Daniel Ocampo Henao,
Manuel Rodríguez Giraldo,
Anderson A. Ruales Barbosa,
Oscar Rodríguez Cifuentes
when the were students of the course. This notes are written in English, because at this level it is expected that any physics student be fluently in reading technical texts in this language.

Some parts are still in Spanish. 

This work have been partially supported by ``Dedicaci\'on Exclusiva 2008-2009''  project: RR 26663

%%% Local Variables: 
%%% mode: latex
%%% TeX-master: "beyond"
%%% End: